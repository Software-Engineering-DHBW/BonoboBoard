\documentclass[a4paper,11pt]{scrartcl}

\usepackage[margin=1in]{geometry}
\usepackage[scaled]{helvet}
\usepackage[T1]{fontenc}
\usepackage[utf8]{inputenc}
\usepackage{amsmath}
\usepackage{mathptmx}
\usepackage{courier}
\usepackage{graphicx}
\usepackage{ulem}
\usepackage{bookmark}
\usepackage{paralist}
\usepackage{ngerman}
\usepackage{fancyhdr}
\usepackage{float}
\usepackage{array}
\usepackage{lipsum}


\graphicspath{ {../img/} }
\renewcommand\familydefault{\sfdefault}




\pagestyle{fancy}
\fancyhf{}
\renewcommand{\headrulewidth}{0pt}
\fancyfoot[C]{\includegraphics[width=\textwidth]{Polygon_gruen}\\ \thepage}


\rhead{\includegraphics[width=\textwidth]{LogoHeader}}
\setlength\headheight{30pt}
\setlength\footskip{15pt}

\begin{document}
\renewcommand*{\arraystretch}{1.2}
\pagenumbering{gobble}
\begin{titlepage}
    \begin{center}
        \vspace*{1cm}\Huge
        \textbf{Dokumentation Template}\par                
        \vspace{0.5cm}\LARGE        
        Software Engineering II\par           
        \vspace{2cm}
        \includegraphics[width=0.5\textwidth]{OptimaLogo_long}\par   
        \vspace{1cm}
        \textbf{Projekttitel: BonoboBoard}\par        
        \vfill\Large   
        Jakob Hutschenreiter (1419081)\\Jiesen Wang (9839152)\\Nick Kramer (3122448)\\Patrick Küsters (2598689)\\Peter Moritz Hinkel (2783930)\par
        %\vspace{2cm}  
        %\includegraphics[width=0.5\textwidth]{Bonobo_Logo}\par        
        \vspace{2cm}
        DHBW Mannheim\\
        \today     
    \end{center}
\end{titlepage}

\section*{Änderungshistorie}
\begin{table}[h]
	\begin{tabular}{@{} p{20mm} p{25mm} p{35mm} p{75mm}}
		\textbf{Revision} & \textbf{Datum} & \textbf{Autor(en)} & \textbf{Beschreibung}\\
		1.0 & 18.01.2022 & NK|PK|JW|MH|JH & Erstellung\\
	\end{tabular}
\end{table}
\noindent
Abkürzungen: Hinzugefügt/Added (A), Änderung/Changed (C), Löschung/Deleted (D)
\vspace{2cm}
\tableofcontents
\newpage
\pagenumbering{arabic}

		%------------------------------------------------------------
		%-----  -----  ----- Begin actual content -----  -----  -----
		%------------------------------------------------------------


\section{Motivation}

Um gemeinsam strukturiert am Projekt arbeiten zu können, setzen wir verschiedene Entwicklertools ein.
Dabei reicht die Spanne von Tools zur Versionsverwaltung hin zu Lern- und Dokumentationstools.

\section{Übersicht der Entwicklertools}

Die Gesamtheit der Entwicklertools wird in verschiedene Kategorien aufgeteilt.
Diese orientieren sich an den, durch uns identifizierten, Haupt-Aufgabengebieten der Projektentwicklung.//

\subsection{Entwicklung}

\begin{table}[H]
\begin{center}
\begin{tabular}{|p{4cm}|p{8cm}|}
\hline
\textbf{Tool} &\textbf{Grund/Einsatzzweck} \\ \hline
Pycharm, Visual Studio Code & Code schreiben \\ \hline
Unittests, Webbrowser & Debuggen\\ \hline
Docker, Anaconda & Abhängigkeiten verwalten\\ \hline
AdobeXD & Softwaredesign (Mockups, Images)\\ \hline
Anaconda & Python-Umgebung verwalten\\ \hline
Git & Versionskontrolle\\ \hline
GitHub & Repository-Verwaltung\\ \hline    
\end{tabular}
\end{center}
\end{table}

\subsection{Test und Automatisierung}

\begin{table}[H]
\begin{center}
\begin{tabular}{|p{4cm}|p{8cm}|}
\hline
\textbf{Tool} &\textbf{Grund/Einsatzzweck} \\ \hline
GitHub Actions & Code testen \\ \hline
Docker & Container bauen\\ \hline
GitHub Actions & Code integrieren\\ \hline
\end{tabular}
\end{center}
\end{table}
 
\subsection{Verbesserung}

\begin{table}[H]
\begin{center}
\begin{tabular}{|p{4cm}|p{8cm}|}
\hline
\textbf{Tool} &\textbf{Grund/Einsatzzweck} \\ \hline
Pycharm, Visual Studio Code & Code formatieren\\ \hline
Pycharm, Visual Studio Code & Refactoring\\ \hline
GitHub, Discord & Code Reviews\\ \hline
\end{tabular}
\end{center}
\end{table}

\subsection{Lernen}

\begin{table}[H]
\begin{center}
\begin{tabular}{|p{4cm}|p{8cm}|}
\hline
\textbf{Tool} &\textbf{Grund/Einsatzzweck} \\ \hline
Youtube, Stackoverflow, Google, Discord & Neues dazulernen\\ \hline
Discord, Screensharing & Wissen teilen\\ \hline
Adobe XD & Neue Designs ausprobieren\\ \hline
OneNote, Discord & Dokumentation neuer Tools/Verfahren\\ \hline
\end{tabular}
\end{center}
\end{table}

\subsection{Ausführen und Verwenden}

\begin{table}[H]
\begin{center}
\begin{tabular}{|p{4cm}|p{8cm}|}
\hline
\textbf{Tool} &\textbf{Grund/Einsatzzweck} \\ \hline
Docker & Projekt bauen und deployen\\ \hline
Server (VPS) & Website bereitstellen\\ \hline
LetsEncrypt & SSL-Verschlüsselung der Website\\ \hline
\end{tabular}
\end{center}
\end{table}


		%------------------------------------------------------------
		%-----  -----  ------ End actual content ------  -----  -----
		%------------------------------------------------------------
\end{document}
































