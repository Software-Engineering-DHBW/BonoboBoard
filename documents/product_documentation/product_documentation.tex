\documentclass[a4paper,11pt]{scrartcl}

\usepackage[margin=1in]{geometry}
\usepackage[scaled]{helvet}
\usepackage[T1]{fontenc}
\usepackage[onehalfspacing]{setspace}
\usepackage[utf8]{inputenc}
\usepackage{amsmath}
\usepackage{mathptmx}
\usepackage{courier}
\usepackage{graphicx}
\usepackage{ulem}
\usepackage{bookmark}
\usepackage{paralist}
\usepackage{ngerman}
\usepackage{fancyhdr}
\usepackage{float}
\usepackage{array}
\usepackage{lipsum}


\graphicspath{ {../img/} }
\renewcommand\familydefault{\sfdefault}




\pagestyle{fancy}
\fancyhf{}
\renewcommand{\headrulewidth}{0pt}
\fancyfoot[C]{\includegraphics[width=\textwidth]{Polygon_gruen}\\ \thepage}


\rhead{\includegraphics[width=\textwidth]{LogoHeader}}
\setlength\headheight{30pt}
\setlength\footskip{15pt}

\begin{document}
\renewcommand*{\arraystretch}{1.2}
\pagenumbering{gobble}
\begin{titlepage}
    \begin{center}
        \vspace*{1cm}\Huge
        \textbf{Produktdokumentation}\par                
        \vspace{0.5cm}\LARGE        
        Software Engineering II\par           
        \vspace{2cm}
        \includegraphics[width=0.5\textwidth]{OptimaLogo_long}\par   
        \vspace{1cm}
        \textbf{Projekttitel: BonoboBoard}\par        
        \vfill\Large   
        Jakob Hutschenreiter (1419081)\\Jiesen Wang (9839152)\\Nick Kramer (3122448)\\Patrick Küsters (9815596)\\Peter Moritz Hinkel (2783930)\par
        %\vspace{2cm}  
        %\includegraphics[width=0.5\textwidth]{Bonobo_Logo}\par        
        \vspace{2cm}
        DHBW Mannheim\\
        \today     
    \end{center}
\end{titlepage}

\section*{Änderungshistorie}
\begin{table}[h]
	\begin{tabular}{@{} p{20mm} p{25mm} p{25mm} p{75mm}}
		\textbf{Revision} & \textbf{Datum} & \textbf{Autor(en)} & \textbf{Beschreibung}\\
		1.0 & 18.03.2022 & NK & A: 1, 2\\
		1.1 & 24.03.2022 & MH & A: 3\\
        1.2 & 26.03.2022 & JH, NK & C: 2\\
	\end{tabular}
\end{table}
\noindent
Abkürzungen: Hinzugefügt/Added (A), Änderung/Changed (C), Löschung/Deleted (D)
\vspace{2cm}
\tableofcontents
\newpage
\pagenumbering{arabic}

		%------------------------------------------------------------
		%-----  -----  ----- Begin actual content -----  -----  -----
		%------------------------------------------------------------
\section{Motivation und Grundlagen}\label{Grundlagen}
Dieses Dokument dient zur Beschreibung der Abläufe, die nötig sind, um das BonoboBoard lokal zu installieren und auszuführen. Auf den Aufbau des Software-Produkts wird hier nicht mehr eingegangen. Bitte ziehen Sie dafür die etwaigen anderen Dokumente heran.\\
Die nachfolgende Beschreibung wurde auf Basis folgender Abhängigkeiten erstellt:
\begin{table}[H]
\begin{tabular}{|p{5cm}|p{5cm}|}
\hline
\textbf{Software/Library} & \textbf{Version} \\ \hline
	Docker & 20.10.12, Build e91ed57 \\ \hline
	Docker Compose & 2.2.3\\ \hline
	Docker Desktop & 4.5.1 (74721)\\ \hline
\end{tabular}
\end{table}
\noindent
Bitte stellen Sie sicher, dass Sie die genannten Voraussetzungen erfüllen, anderweitig kann nicht sichergestellt werden, dass die Installation auf Ihrem System ordnungsgemäß funktioniert.\\

\noindent Wenn Sie das Produkt lediglich nutzen möchten, können Sie die Installationsdokumentation überspringen und direkt zu Abschnitt \ref{Kurzanleitung} wechseln. Kein Nutzer muss das BonoboBoard lokal installieren, die aktuelle Version kann immer unter \url{https://bonoboboard.de/} gefunden und genutzt werden. Falls Sie das Produkt weiterentwickeln möchten oder eine lokale Installation anstreben, ist mit Abschnitt \ref{Installationsdokumentation} fortzufahren. 

\section{Installationsdokumentation}\label{Installationsdokumentation}
Durch die Nutzung von Containern unter Docker lässt sich die Installation in einigen wenigen Schritten behandeln.
\subsection{Beziehen des Source-Code}
Der Source-Code wird auf GitHub gepflegt\footnote{https://github.com/Software-Engineering-DHBW/BonoboBoard}. Da es sich um ein öffentliches Repository handelt, kann der Code ohne weitere Authentifizierung lokal geklont werden. Für eine detaillierte Anleitung des Klon-Prozesses wird auf die offizielle Dokumentation von GitHub verwiesen\footnote{https://docs.github.com/en/repositories/creating-and-managing-repositories/cloning-a-repository}.\\
Ist das Klonen abgeschlossen, sollte folgende Struktur auf der ersten Ebene des Projekts zu finden sein:
\begin{figure}[H]
\begin{center}
\includegraphics[width=0.8\textwidth]{folder_repo_1}
\caption{Ordnerstruktur des heruntergeladenen Projekts}
\label{img:folder_1}
\end{center}
\end{figure}

\subsection{Installation und Start der Docker Container}
%TODO maybe Änderung des Ports falls 80 belegt?
Dieser Abschnitt beschreibt den Ablauf, wie die Docker-Umgebung aufgebaut und gestartet wird.
Dazu muss in den Ordner \textit{bonobo-board} (siehe dritter Ordner von oben in Abbildung \ref{img:folder_1}) gewechselt werden.
In diesem befindet sich eine \textit{docker-compose.yml} Datei und zwei weitere Dockerfiles: \textit{Dockerfile.base}, \textit{Dockerfile.django}.
Sollten diese nicht vorhanden sein, ist entweder nicht der richtige Ordner ausgewählt oder beim Herunterladen der Dateien sind Fehler aufgetreten.
Hier eine Abbildung des Verzeichnisses zum Vergleich:
\begin{figure}[H]
\begin{center}
\includegraphics[width=0.8\textwidth]{folder_repo_2}
\caption{Struktur in \textit{bonobo-board}}
\label{img:folder_1}
\end{center}
\end{figure}

\noindent Nun muss ein Terminal/PowerShell in diesem Ordner gestartet werden.\\
Im Terminal gibt es nun mehrere Wege die Container zum Laufen zu bekommen.
Der einfachste Weg stellt das Ausführen der bereitgestellten Skripte dar.

\subsubsection*{Automatische Installation Windows}
Unter Windows muss das PowerShell-Skript namens \textit{build\_image.ps1} ausgeführt werden.
Der Befehl hierfür lautet \texttt{.\textbackslash build\_image.ps1 -a}, welcher die beiden benötigten Docker Images erstellt.

\subsubsection*{Automatische Installation Linux}
Unter \textbf{Ubuntu (Linux)} ist das Bash-Skript \textit{build\_image} auszuführen.
Der Befehl hierfür ist\\
\texttt{./build\_image -a}, welcher die Docker Images erstellt.\\

\noindent Nach dem Erstellen der Docker Images reicht ein simples \texttt{docker-compose up} und
die Container werden erstellt. Alternativ kann der Befehl \texttt{docker-compose up -d} ausgeführt werden,
wenn die Kommandozeile nach dem Ausführen des Befehls nicht blockiert sein soll.
In diesem Fall laufen die gestarteten Container im Hintergrund weiter.
Abbildung \ref{img:docker-compose} zeigt beispielhaft die Ausgabe der Befehle in der Kommandozeile.\\ 
Bei Fehlermeldungen ist sicherzustellen, dass der docker-daemon läuft (Windows: läuft Docker-Desktop?) oder alle Voraussetzungen (siehe \ref{Grundlagen}) erfüllt sind.
\begin{figure}[H]
\begin{center}
\includegraphics[width=0.8\linewidth]{docker-compose}
\caption{docker-compose in der Kommandozeile}
\label{img:docker-compose}
\end{center}
\end{figure}

\subsection{Darstellung im Webbrowser}
Da die Container nun gestartet sind, ist zu prüfen ob alles funktioniert und die Website lokal erreichbar ist.
Dafür ist ein Webbrowser nach Wahl zu öffnen und \url{http://localhost:80/} einzugeben.
Es sollte sich die Website, wie in Abbildung \ref{img:darstellung} dargestellt, öffnen.
Damit ist die lokale Installation des BonoboBoard abgeschlossen.
\begin{figure}[H]
\begin{center}
\includegraphics[width=0.8\textwidth]{webbrowser_localhost}
\caption{Darstellung im Browser}
\label{img:darstellung}
\end{center}
\end{figure}
\subsection{Installationsprobleme}
Haben alle Schritte vorab funktioniert und die Darstellung funktioniert trotzdem nicht, ist sicherzustellen, dass kein anderer Dienst Port 80 belegt.\\
\\
Unter \textbf{Windows} kann dies mit Hilfe der Kommandozeile und des Taskmanagers überprüft werden.
Dazu ist der Befehl \texttt{netstat -ano -t tcp} auszuführen und die Zeile zu lokalisieren,
in der die lokale Adresse 0.0.0.0:80 durch die remote Adresse 0.0.0.0:0 abgehört wird.
Die PID am Ende dieser Zeile kann im Task-Manager auf eine Anwendung zurückgeführt werden.
In Abbildung \ref{img:netstat} ist dies dargestellt. In diesem Fall wird die Website ordnungsgemäß dargestellt.
Benutzt ein anderer Prozess (außer Docker-Desktop) diesen Port, ist dieser zu terminieren.
\begin{figure}[H]
\begin{center}
\includegraphics[width=0.75\textwidth]{netstat_port80}
\caption{Überprüfung des lokalen Port 80}
\label{img:netstat}
\end{center}
\end{figure}

\noindent Unter \textbf{Ubuntu (Linux)} ist die Abfrage durch \texttt{sudo netstat -peeanut | grep :80} möglich.
Wird in der Kommandozeile ein Service aufgelistet, wird Port 80 bereits genutzt. Die angegebene PID in der letzten Spalte vor dem Servicenamen (\textit{\textbf{25338}/docker-proxy}) kann dazu genutzt werden, den Prozess mittels \texttt{kill <PID>} zu terminieren.\\
Als letztes Mittel kann der Port in der \textit{docker-compose.yml} Datei angepasst werden, um das BonoboBoard auf einem von 80 verschiedenen Port zu starten.

\clearpage
\section{Kurzanleitung}\label{Kurzanleitung}
Herzlich willkommen beim BonoboBoard! Danke, dass Sie sich für ein Produkt von Optima Connect entschieden haben. 

\bigskip
\noindent In dieser Kurzanleitung werden wir sie mit den grundlegenden Funktionen Ihres persönlichen DHBW-Dashboards vertraut machen, sodass Sie bereits in Kürze Ihren Workflow mit BonoboBoard optimieren können.

\subsection{Anmeldung}
Um das BonobobBoard benutzen zu können, gehen sie im Webbrowser Ihrer Wahl auf \url{https://bonoboboard.de} oder - falls Sie eine lokale Installation verwenden - auf \url{http://localhost:80/}. Von dort aus gelangen Sie zum Login (siehe Abb. \ref{img:login}).

\begin{figure}[H]
	\begin{center}
		\includegraphics[width=0.8\textwidth]{login}
		\caption{BonoboBoard - Login}
		\label{img:login}
	\end{center}
\end{figure}

\noindent Geben Sie nun dort Ihre DHBW-Mailadresse (\frqq{}S-Adresse\flqq{}) und Ihr Passwort, dass sie für Ihren DHBW-Account verwenden, ein. Keine Sorge - Ihr Passwort wird weder lokal noch auf unserem Server gespeichert. Auch nicht als Hash. Wir verwenden es lediglich, um eine Verbindung, mit den DHBW-Services aufzubauen. Wird diese Verbindung geschlossen, müssen sie sich allerdings neu anmelden.

\bigskip
\noindent Anschließend geben Sie noch Ihre Kursbezeichnung an. Beispielweise: \frqq{}TINF19 IT2\flqq{}. Achten Sie bitte auf das Leerzeichen zwischen Semester- und Kursbezeichnung. Oder verwenden Sie einfach die Autovervollständigung (siehe Abb. \ref{img:autocomplete}).
\begin{figure}[H]
	\begin{center}
		\includegraphics[width=0.8\textwidth]{autocomplete}
		\caption{BonoboBoard - Kusrauswahl}
		\label{img:autocomplete}
	\end{center}
\end{figure}
\noindent

\subsection{Home}
Nach erfolgreichem Login gelangen Sie auf das Dashboard (siehe Abb. \ref{img:dashboard}). Dort können alle Termine für den aktuellen Tag, sowie Ihre aktuelle Gesamtnote einsehen.

\begin{figure}[H]
	\begin{center}
		\includegraphics[width=0.8\textwidth]{Dashboard}
		\caption{BonoboBoard - Dashboard}
		\label{img:dashboard}
	\end{center}
\end{figure}
\noindent


\subsection{Seitennavigation}
Das BonoboBoard lässt sich ganz einfach über die Sidebar bedienen (siehe Abb. \ref{img:sidebar}). Von dort aus können alle wichtigen Funktionen des BonoboBoard mit nur einem Klick erreicht werden. Außerdem können Sie sich über den Button \frqq{}Logout\flqq{} am unteren Rand der Sidebar abmelden.

\begin{figure}[H]
	\begin{center}
		\includegraphics[width=0.15\textwidth]{sidebar}
		\caption{BonoboBoard - Seitennavigation}
		\label{img:sidebar}
	\end{center}
\end{figure}

\noindent Die Sidebar kann mit einem Klick eingeklappt und wieder verankert werden.

\subsection{Vorlesungsplan}
Der Vorlesungsplan zeigt all Ihre Veranstaltungen der aktuellen Kalenderwoche an. Die Vorlesungstermine werden in einer Zeitleiste von 7:00 bis 21:45 zusammen mit ihren Titel, der Veranstaltungszeit und der Raumnummer (falls vorhanden) angezeigt (siehe Abb. \ref{img:lecture}). 

\begin{figure}[H]
	\begin{center}
		\includegraphics[width=0.8\textwidth]{lecture_plan}
		\caption{BonoboBoard - Vorlesungsplan}
		\label{img:lecture}
	\end{center}
\end{figure}

\noindent Über die Beiden Pfeile in der Kopfleiste (siehe Abb. \ref{img:lecture_nav}) gelangen Sie zum Vorlesungsplan der vergangenen oder folgenden Wochen. Wenn sie auf den Button \frqq{}Heute\flqq{} tippen, gelangen sie zur aktuellen Woche zurück.

\begin{figure}[H]
	\begin{center}
		\includegraphics[width=0.8\textwidth]{lecture_nav}
		\caption{BonoboBoard - Vorlesungsplan Navigation}
		\label{img:lecture_nav}
	\end{center}
\end{figure}

\noindent Die angezeigten Veranstaltungen werden von der Website \frqq{}\href{https://vorlesungsplan.dhbw-mannheim.de/index.php}{DHBW - Kurskalender}\flqq{} bezogen.

\bigskip
Mit einem Klick auf eine der angezeigten Veranstaltungen öffnet sich ein Dialog, in dem ein Link für den jeweiligen Online Vorlesungsraum hinterlegt werden kann. Dort können Sie einen Link aus einer Liste ausgewählt werden, die direkt aus dem DHBW Moodle System bezogen wird (siehe Abb. \ref{img:Moodle_link}).

\begin{figure}[H]
	\begin{center}
		\includegraphics[width=0.5\textwidth]{Moodle_link}
		\caption{BonoboBoard - Kursraumlink hinzufügen}
		\label{img:Moodle_link}
	\end{center}
\end{figure}

\noindent Anschließend können Sie mit einem Klick auf \frqq{}Zum Kursraum\flqq{} direkt in den Online Vorlesungsraum gelangen \ref{img:Kursraum}).

\begin{figure}[H]
	\begin{center}
		\includegraphics[width=0.4\textwidth]{Kursraum}
		\caption{BonoboBoard - Zum Kursraum}
		\label{img:Kursraum}
	\end{center}
\end{figure}

\subsection{Leistungsübersicht}
In der Leistungsübersicht werden Ihre Noten angezeigt, die von Ihrem Dualis Account bezogen werden. Auf der linken Seite sind unter \frqq{}Noten\flqq{} die Einzelnoten der jeweiligen Fächer vermerkt. Rechts finden Sie unter \frqq{}GPA\flqq{} Ihre derzeitige Gesamtnote (siehe Abb. \ref{img:lecture_nav})

\begin{figure}[H]
	\begin{center}
		\includegraphics[width=0.8\textwidth]{dualis}
		\caption{BonoboBoard - Lesitungsübersicht}
		\label{img:dualis}
	\end{center}
\end{figure}
\noindent

\subsection{E-Mail}
\label{sec:email}
Das BonoboBoard ermöglicht es Ihnen schnell und einfach eine E-Mail von Ihrem Studierenden-Mailkonto (\frqq{}S-Adresse\flqq{}) zu verschicken. Navigieren sie dazu auf der Sidebar zu \frqq{}Emails\flqq{} oder klicken Sie im Dashboard auf \frqq{}Email schreiben\flqq{}. Dort können sie dann Empfänger, Betreff und Nachrichtentext eingeben und die E-Mail dann verschicken (siehe Abb. \ref{img:mail}).
 
\begin{figure}[H]
	\begin{center}
		\includegraphics[width=0.8\textwidth]{Mail}
		\caption{BonoboBoard - Email}
		\label{img:mail}
	\end{center}
\end{figure}

\noindent Die E-Mail wird von Ihrem DHBW-Mailkonto verschickt und wird Ihnen somit auch in Ihrem Online Postfach als gesendete Mail angezeigt.

Das BonoboBoard zeigt Ihnen zudem alle in Zimbra angelegten Kontakte an. Mit einem Klick auf einen der Buttons neben Ihren Kontakten, können Sie dessen Email-Adresse als Empfänger, Cc oder Bcc hinzufügen.

\section{Danksagung}
Das Team von Optimal Connect wünscht viel Vergnügen mit dem BonoboBoard. Für Fragen und Anregungen oder bei etwaigen Fehlern, melden Sie sich gern via GitHub: \url{https://github.com/Software-Engineering-DHBW/BonoboBoard} bei uns.
		%------------------------------------------------------------
		%-----  -----  ------ End actual content ------  -----  -----
		%------------------------------------------------------------
\end{document}