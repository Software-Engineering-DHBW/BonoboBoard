\documentclass[a4paper,11pt]{scrartcl}

\usepackage[margin=1in]{geometry}
\usepackage[scaled]{helvet}
\usepackage[T1]{fontenc}
\usepackage[utf8]{inputenc}
\usepackage{amsmath}
\usepackage{mathptmx}
\usepackage{courier}
\usepackage{graphicx}
\usepackage{ulem}
\usepackage{bookmark}
\usepackage{paralist}
\usepackage{ngerman}
\usepackage{fancyhdr}
\usepackage{float}
\usepackage{array}
\usepackage{lipsum}


\graphicspath{ {../img/} }
\renewcommand\familydefault{\sfdefault}




\pagestyle{fancy}
\fancyhf{}
\renewcommand{\headrulewidth}{0pt}
\fancyfoot[C]{\includegraphics[width=\textwidth]{Polygon_gruen}\\ \thepage}


\rhead{\includegraphics[width=\textwidth]{LogoHeader}}
\setlength\headheight{30pt}
\setlength\footskip{15pt}

\begin{document}
\renewcommand*{\arraystretch}{1.2}
\pagenumbering{gobble}
\begin{titlepage}
    \begin{center}
        \vspace*{1cm}\Huge
        \textbf{Software Design Dokumentation}\par                
        \vspace{0.5cm}\LARGE        
        Software Engineering II\par           
        \vspace{2cm}
        \includegraphics[width=0.5\textwidth]{OptimaLogo_long}\par   
        \vspace{1cm}
        \textbf{Projekttitel: BonoboBoard}\par        
        \vfill\Large   
        Jakob Hutschenreiter (1419081)\\Jiesen Wang (9839152)\\Nick Kramer (3122448)\\Patrick Küsters (2598689)\\Peter Moritz Hinkel (2783930)\par
        %\vspace{2cm}  
        %\includegraphics[width=0.5\textwidth]{Bonobo_Logo}\par        
        \vspace{2cm}
        DHBW Mannheim\\
        \today     
    \end{center}
\end{titlepage}

\section*{Änderungshistorie}
\begin{table}[h]
	\begin{tabular}{@{} p{20mm} p{25mm} p{25mm} p{75mm}}
		\textbf{Revision} & \textbf{Datum} & \textbf{Autor(en)} & \textbf{Beschreibung}\\
		1.0 & 24.01.2022 & JW & A: 1, 3 \\ 
		1.1 & 25.01.2022 & JW|PH|NK & A: 2  \\ 
		1.2 & 25.01.2022 & NK & C: 2, A: 4, 5  \\ 
	\end{tabular}
\end{table}
\noindent
Abkürzungen: Hinzugefügt/Added (A), Änderung/Changed (C), Löschung/Deleted (D)
\vspace{2cm}
\tableofcontents
\newpage
\pagenumbering{arabic}

		%------------------------------------------------------------
		%-----  -----  ----- Begin actual content -----  -----  -----
		%------------------------------------------------------------



\section{Einleitung}
Um sicher zu gehen, dass alle Software Design Ziele des BonoboBoard-Projekts richtig umgesetzt werden, werden diese in diesem Dokument festgehalten. Das Dokument erleichtert die Analyse, Planung, Implementierung und Entscheidungsfindung. Anders als im Anforderungsdokument wird nun hier beschrieben wie die Ziele umgesetzt werden. 


\section{Entwicklersetup}
Verwendete Software und Libiaries Versionen:
\begin{table}[H]
\begin{tabular}{|p{4cm}|p{8cm}|}
\hline
\textbf{Software/Libary} & \textbf{Version} \\ \hline
	Django &  4.0.2\\ \hline
	Django Tables & 2.4.1 \\ \hline
	dj-database-url & 0.5.0 \\ \hline
	dj-database-url & 0.5.0 \\ \hline
	Gunicorn & 20.1.0 \\ \hline
	Docker & 20.10.12, Build e91ed57 \\ \hline
	PyCharm & 2021.3.2 \\ \hline
	VisualStudio Code & 1.64.2 \\ \hline
	Beautiful Soup & 4.8.2 \\ \hline
	Requests & 2.27.1 \\ \hline
	iCalendar & 4.0.9 \\ \hline
	Pandas & 1.4.0 \\ \hline
	lxml & 4.5.0 \\ \hline
	SQLAlchemy & 1.4.31 \\ \hline
	asgiref & 3.4.1 \\ \hline
	autopep8 & 1.6.0 \\ \hline
	Python & 3.9 \\ \hline
	pycodestyle & 2.8.0 \\ \hline
	python-decouple & 3.5 \\ \hline
	pytz & 2021.3 \\ \hline
	sqlparse & 0.4.2 \\ \hline
	toml & 0.10.2 \\ \hline
	Unipath & 1.1 \\ \hline
	whitenoise & 5.3.0 \\ \hline
\end{tabular}
\end{table}

\section{Verwendete Tools}
Für ein gemeinsames strukturiertes Arbeiten am Projekt, werden je nach Aufgabe verschiedene Tools eingesetzt. 
	\subsection{Tools für das Projektmanagement/Organisation}
\begin{table}[H]
\begin{tabular}{|p{4cm}|p{8cm}|}
\hline
\textbf{Tool} &\textbf{Einsatzzweck} \\ \hline
Discord &  Austausch von Nachrichten, Virtuelle Meetings, Umfragen, Informationsmanagement\\ \hline
WhatsApp &  Austausch von Nachrichten\\ \hline
DropBox & Informationsmanagement \\ \hline
Google Kalender & Terminplanung   \\ \hline
Jira &  Aufgaben- und Projektmanagement, Prozessmanagement \\ \hline
Latex und Git &  Dokumente erstellen und bearbeiten \\ \hline
OneNote &  Informationsmanagement, Dokumentvorlagen erstellen und bearbeiten \\ \hline
Draw.io &  Diagramme erstellen und bearbeiten \\ \hline
Git und GitHub & Versionskontrolle und Repository-Verwaltung \\ \hline
Adobe XD & Ideen Entwicklung, Erstellung von Mockups \\ \hline
\end{tabular}
\end{table}
	\subsection{Tools für das Projektmanagement/Organisation}
Alle verwendeten Tools für die Softwareentwicklung sind aus dem seperaten Dokument: \glqq Tools für die Softwareentwicklung\grqq{} zu entnehmen.

\section{Zentrale Designentscheidung}

\section{Zusammenfassung}



		%------------------------------------------------------------
		%-----  -----  ------ End actual content ------  -----  -----
		%------------------------------------------------------------
\end{document}