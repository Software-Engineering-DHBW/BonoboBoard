\documentclass[a4paper,11pt]{scrartcl}

\usepackage[margin=1in]{geometry}
\usepackage[scaled]{helvet}
\usepackage[T1]{fontenc}
\usepackage[utf8]{inputenc}
\usepackage{amsmath}
\usepackage{mathptmx}
\usepackage{courier}
\usepackage{graphicx}
\usepackage{ulem}
\usepackage{bookmark}
\usepackage{paralist}
\usepackage{ngerman}
\usepackage{fancyhdr}
\usepackage{float}
\usepackage{array}
\usepackage{lipsum}


\graphicspath{ {../img/} }
\renewcommand\familydefault{\sfdefault}




\pagestyle{fancy}
\fancyhf{}
\renewcommand{\headrulewidth}{0pt}
\fancyfoot[C]{\includegraphics[width=\textwidth]{Polygon_gruen}\\ \thepage}


\rhead{\includegraphics[width=\textwidth]{LogoHeader}}
\setlength\headheight{30pt}
\setlength\footskip{15pt}

\begin{document}
\renewcommand*{\arraystretch}{1.2}
\pagenumbering{gobble}
\begin{titlepage}
    \begin{center}
        \vspace*{1cm}\Huge
        \textbf{Software Design Dokumentation}\par                
        \vspace{0.5cm}\LARGE        
        Software Engineering II\par           
        \vspace{2cm}
        \includegraphics[width=0.5\textwidth]{OptimaLogo_long}\par   
        \vspace{1cm}
        \textbf{Projekttitel: BonoboBoard}\par        
        \vfill\Large   
        Jakob Hutschenreiter (1419081)\\Jiesen Wang (9839152)\\Nick Kramer (3122448)\\Patrick Küsters (2598689)\\Peter Moritz Hinkel (2783930)\par
        %\vspace{2cm}  
        %\includegraphics[width=0.5\textwidth]{Bonobo_Logo}\par        
        \vspace{2cm}
        DHBW Mannheim\\
        \today     
    \end{center}
\end{titlepage}

\section*{Änderungshistorie}
\begin{table}[h]
	\begin{tabular}{@{} p{20mm} p{25mm} p{25mm} p{75mm}}
		\textbf{Revision} & \textbf{Datum} & \textbf{Autor(en)} & \textbf{Beschreibung}\\
		1.0 & 24.01.2022 & JW & Erstellung\\ 
	\end{tabular}
\end{table}
\noindent
Abkürzungen: Hinzugefügt/Added (A), Änderung/Changed (C), Löschung/Deleted (D)
\vspace{2cm}
\tableofcontents
\newpage
\pagenumbering{arabic}

		%------------------------------------------------------------
		%-----  -----  ----- Begin actual content -----  -----  -----
		%------------------------------------------------------------



\section{Einleitung}
	\subsection{Motivation} % Motivation, Thema, Ziel
Um sicher zu gehen, dass alle Software Design Ziele richtig umgesetzt werden, werden diese in diesem Dokument festgehalten.

\section{Hauptteil}
	\subsection{Entwicklersetup}

	\subsection{Verwendete Tools}
Für ein gemeinsames strukturiertes Arbeiten am Projekt, werden je nach Aufgabe verschiedene Tools eingesetzt. 
		\subsubsection{Tools für das Projektmanagement/Organisation}
\begin{table}[H]
\begin{tabular}{|p{4cm}|p{8cm}|}
\hline
\textbf{Tool} &\textbf{Einsatzzweck} \\ \hline
Discord &  Austausch von Nachrichten, Virtuelle Meetings, Umfragen, Informationsmanagement\\ \hline
WhatsApp &  Austausch von Nachrichten\\ \hline
DropBox & Informationsmanagement \\ \hline
Google Kalender & Terminplanung   \\ \hline
Jira &  Aufgaben- und Projektmanagement, Prozessmanagement \\ \hline
Latex und Git &  Dokumente erstellen und bearbeiten \\ \hline
OneNote &  Informationsmanagement, Dokumentvorlagen erstellen und bearbeiten \\ \hline
Draw.io &  Diagramme erstellen und bearbeiten \\ \hline
Git und GitHub & Versionskontrolle und Repository-Verwaltung \\ \hline
Adobe XD & Ideen Entwicklung, Erstellung von Mockups \\ \hline
\end{tabular}
\end{table}
		\subsubsection{Tools für das Projektmanagement/Organisation}
Alle verwendeten Tools für die Softwareentwicklung sind aus dem seperaten Dokument: \glqq Tools für die Softwareentwicklung\grqq{} zu entnehmen.

	\subsection{Zentrale Designentscheidung}



		%------------------------------------------------------------
		%-----  -----  ------ End actual content ------  -----  -----
		%------------------------------------------------------------
\end{document}