\documentclass[a4paper,11pt]{scrartcl}

\usepackage[margin=1in]{geometry}
\usepackage[scaled]{helvet}
\usepackage[T1]{fontenc}
\usepackage[utf8]{inputenc}
\usepackage{amsmath}
\usepackage{mathptmx}
\usepackage{courier}
\usepackage{graphicx}
\usepackage{ulem}
\usepackage{bookmark}
\usepackage{paralist}
\usepackage{ngerman}
\usepackage{fancyhdr}
\usepackage{float}
\usepackage{array}


\graphicspath{ {../img/} }
\renewcommand\familydefault{\sfdefault}




\pagestyle{fancy}
\fancyhf{}
\renewcommand{\headrulewidth}{0pt}
\fancyfoot[C]{\includegraphics[width=\textwidth]{Polygon_gruen}\\ \thepage}


\rhead{\includegraphics[width=\textwidth]{LogoHeader}}
\setlength\headheight{30pt}
\setlength\footskip{15pt}

\begin{document}
\renewcommand*{\arraystretch}{1.2}
\pagenumbering{gobble}
\begin{titlepage}
    \begin{center}
        \vspace*{1cm}\Huge
        \textbf{Anforderungsanalyse}\par                
        \vspace{0.5cm}\LARGE        
        Software Engineering II\par           
        \vspace{2cm}
        \includegraphics[width=0.5\textwidth]{OptimaLogo_long}\par   
        \vspace{1cm}
        \textbf{Projekttitel: BonoboBoard}\par        
        \vfill\Large   
        Jakob Hutschenreiter (1419081)\\Jiesen Wang (9839152)\\Nick Kramer (3122448)\\Patrick Küsters (2598689)\\Peter Moritz Hinkel (2783930)\par
        %\vspace{2cm}  
        %\includegraphics[width=0.5\textwidth]{Bonobo_Logo}\par        
        \vspace{2cm}
        DHBW Mannheim\\
        \today     
    \end{center}
\end{titlepage}

\section*{Änderungshistorie}
\begin{table}[h]
	\begin{tabular}{@{} p{20mm} p{25mm} p{25mm} p{75mm}}
		\textbf{Revision} & \textbf{Datum} & \textbf{Autor(en)} & \textbf{Beschreibung}\\
		1.0 & 31.01.2022 & NK|PK & Erstellung, A: 1.1, 1.2, 1.3\\
		1.1 & 31.01.2022 & JW & A: 
	\end{tabular}
\end{table}
\noindent
Abkürzungen: Hinzugefügt/Added (A), Änderung/Changed (C), Löschung/Deleted (D)
\vspace{2cm}
\tableofcontents
\newpage
\pagenumbering{arabic}

		%------------------------------------------------------------
		%-----  -----  ----- Begin actual content -----  -----  -----
		%------------------------------------------------------------


\section{Einleitung}
	\subsection{Motivation} % Motivation, Thema, Ziel
BonoboBoard ist ein kostenfreier webbasierter Service für alle Studierenden der DHBW Mannheim, die statt vieler unabhängiger Websites eine einzige Übersicht aller auf die Hochschule bezogenen Inhalte erhalten wollen.\\
Es stellt Funktionen bereit, die alle relevanten Websites der DHBW Mannheim nach Informationen durchsucht und diese in Form eines Dashboards darstellt.\\
Da keine Konkurrenzprodukte existieren, ist unser Produkt am Markt einzigartig.


	\subsection{Zielgruppe} % Personas
Das BonoboBoard soll Studierende der DHBW Mannheim ansprechen. Da sich die Anforderungsdefinition nach der Persona Methode richtet, folgt eine Beschreibung unserer Persona:\\\\
Hans ist 20 Jahre alt, hat sein Abitur absolviert und studiert nun an der DHBW Mannheim im zweiten Semester Informationstechnik. Er ist sehr motiviert und möchte sein Studium bestmöglich absolvieren. Dabei nutzt er Tools, die es ihm erlauben effektiver zu arbeiten. Beispielsweise nutzt er einen Kalender, um seine Termine mit dem Vorlesungsplan in Einklang zu bringen.\\ % automatisches / manuelles Eintragen ? Passt hier nicht wirklich rein
\noindent
Da er sich im zweiten Semester befindet, hat er bereits alle Websites der DHBW Mannheim kennengelernt. Gerade bei der Verwaltung und Struktur der DHBW-Plattform Moodle und Dualis sieht er einige Schwachstellen und sucht nach übersichtlichen Alternativen, die den Informationsfluss der DHBW auf einem Kanal bündeln. Er begibt sich auf die Suche nach Tools, die ihm den Universitätsalltag erleichtern.\\
Beim Arbeiten mit Tools legt er wert darauf, dass diese intuitiv zu bedienen sind und ein halbwegs ansprechendes Design vorweisen. Daher ist er von den Websites, mit welchen er im DHBW Studium arbeiten muss, frustriert.



% Auf der Suche nach einem weiteren Tool, was ihm die Arbeit erleichtert stößt er auf das BonoboBoard.

	\subsection{Detaillierte Ziele} % Szenarios

\textbf{Szenario 1: Den DHBW Vorlesungsplan einsehen}\par\noindent
Hans benutzt das BonoboBoard, um seinen wöchentlichen Vorlesungsplan direkt auf seinem Dashboard angezeigt zu bekommen. Durch das Auswählen einer bestimmten Vorlesung kann er sehen, welchen Vorlesungsraum, er betreten muss. Dies ist unabhängig davon, ob die Veranstaltung online oder in Präsenz stattfindet. Hierbei hilft ihm vor allem die einfache Zuordnung der Links zu den Vorlesungsräumen für die unterschiedlichen Online-Plattformen, wie Microsoft-Teams oder BigBlueButton, welche sonst mühsam zu ermitteln war. Sollte diese automatische Zuordnung einmal nicht möglich sein, kann Hans für zukünftige Termine Vorlesungslinks zu Räumen ergänzen. Die Veranstaltungen nehmen ihrer Dauer entsprechend Platz im Kalender ein, wodurch Hans eine Übersicht über ihm frei zur Verfügung stehende Zeiträume erhält. Zurückliegende Veranstaltungen werden farbig unterschiedlich zu noch ausstehenden hervorgehoben, wodurch Hans weiß, welche Termine für ihn noch ausstehen.


\section{Hauptteil}
	\subsection{Anforderungen} % User Stories
	
	\subsection{Funktionalität} % Features
	
\section{Zusammenfassung}



		%------------------------------------------------------------
		%-----  -----  ------ End actual content ------  -----  -----
		%------------------------------------------------------------
\end{document}

























